\documentclass[]{article}
\usepackage{lmodern}
\usepackage{amssymb,amsmath}
\usepackage{ifxetex,ifluatex}
\usepackage{fixltx2e} % provides \textsubscript
\ifnum 0\ifxetex 1\fi\ifluatex 1\fi=0 % if pdftex
  \usepackage[T1]{fontenc}
  \usepackage[utf8]{inputenc}
\else % if luatex or xelatex
  \ifxetex
    \usepackage{mathspec}
  \else
    \usepackage{fontspec}
  \fi
  \defaultfontfeatures{Ligatures=TeX,Scale=MatchLowercase}
\fi
% use upquote if available, for straight quotes in verbatim environments
\IfFileExists{upquote.sty}{\usepackage{upquote}}{}
% use microtype if available
\IfFileExists{microtype.sty}{%
\usepackage{microtype}
\UseMicrotypeSet[protrusion]{basicmath} % disable protrusion for tt fonts
}{}
\usepackage[margin=1in]{geometry}
\usepackage{hyperref}
\hypersetup{unicode=true,
            pdftitle={Homework 0},
            pdfauthor={YOUR NAME HERE},
            pdfborder={0 0 0},
            breaklinks=true}
\urlstyle{same}  % don't use monospace font for urls
\usepackage{color}
\usepackage{fancyvrb}
\newcommand{\VerbBar}{|}
\newcommand{\VERB}{\Verb[commandchars=\\\{\}]}
\DefineVerbatimEnvironment{Highlighting}{Verbatim}{commandchars=\\\{\}}
% Add ',fontsize=\small' for more characters per line
\usepackage{framed}
\definecolor{shadecolor}{RGB}{248,248,248}
\newenvironment{Shaded}{\begin{snugshade}}{\end{snugshade}}
\newcommand{\AlertTok}[1]{\textcolor[rgb]{0.94,0.16,0.16}{#1}}
\newcommand{\AnnotationTok}[1]{\textcolor[rgb]{0.56,0.35,0.01}{\textbf{\textit{#1}}}}
\newcommand{\AttributeTok}[1]{\textcolor[rgb]{0.77,0.63,0.00}{#1}}
\newcommand{\BaseNTok}[1]{\textcolor[rgb]{0.00,0.00,0.81}{#1}}
\newcommand{\BuiltInTok}[1]{#1}
\newcommand{\CharTok}[1]{\textcolor[rgb]{0.31,0.60,0.02}{#1}}
\newcommand{\CommentTok}[1]{\textcolor[rgb]{0.56,0.35,0.01}{\textit{#1}}}
\newcommand{\CommentVarTok}[1]{\textcolor[rgb]{0.56,0.35,0.01}{\textbf{\textit{#1}}}}
\newcommand{\ConstantTok}[1]{\textcolor[rgb]{0.00,0.00,0.00}{#1}}
\newcommand{\ControlFlowTok}[1]{\textcolor[rgb]{0.13,0.29,0.53}{\textbf{#1}}}
\newcommand{\DataTypeTok}[1]{\textcolor[rgb]{0.13,0.29,0.53}{#1}}
\newcommand{\DecValTok}[1]{\textcolor[rgb]{0.00,0.00,0.81}{#1}}
\newcommand{\DocumentationTok}[1]{\textcolor[rgb]{0.56,0.35,0.01}{\textbf{\textit{#1}}}}
\newcommand{\ErrorTok}[1]{\textcolor[rgb]{0.64,0.00,0.00}{\textbf{#1}}}
\newcommand{\ExtensionTok}[1]{#1}
\newcommand{\FloatTok}[1]{\textcolor[rgb]{0.00,0.00,0.81}{#1}}
\newcommand{\FunctionTok}[1]{\textcolor[rgb]{0.00,0.00,0.00}{#1}}
\newcommand{\ImportTok}[1]{#1}
\newcommand{\InformationTok}[1]{\textcolor[rgb]{0.56,0.35,0.01}{\textbf{\textit{#1}}}}
\newcommand{\KeywordTok}[1]{\textcolor[rgb]{0.13,0.29,0.53}{\textbf{#1}}}
\newcommand{\NormalTok}[1]{#1}
\newcommand{\OperatorTok}[1]{\textcolor[rgb]{0.81,0.36,0.00}{\textbf{#1}}}
\newcommand{\OtherTok}[1]{\textcolor[rgb]{0.56,0.35,0.01}{#1}}
\newcommand{\PreprocessorTok}[1]{\textcolor[rgb]{0.56,0.35,0.01}{\textit{#1}}}
\newcommand{\RegionMarkerTok}[1]{#1}
\newcommand{\SpecialCharTok}[1]{\textcolor[rgb]{0.00,0.00,0.00}{#1}}
\newcommand{\SpecialStringTok}[1]{\textcolor[rgb]{0.31,0.60,0.02}{#1}}
\newcommand{\StringTok}[1]{\textcolor[rgb]{0.31,0.60,0.02}{#1}}
\newcommand{\VariableTok}[1]{\textcolor[rgb]{0.00,0.00,0.00}{#1}}
\newcommand{\VerbatimStringTok}[1]{\textcolor[rgb]{0.31,0.60,0.02}{#1}}
\newcommand{\WarningTok}[1]{\textcolor[rgb]{0.56,0.35,0.01}{\textbf{\textit{#1}}}}
\usepackage{longtable,booktabs}
\usepackage{graphicx,grffile}
\makeatletter
\def\maxwidth{\ifdim\Gin@nat@width>\linewidth\linewidth\else\Gin@nat@width\fi}
\def\maxheight{\ifdim\Gin@nat@height>\textheight\textheight\else\Gin@nat@height\fi}
\makeatother
% Scale images if necessary, so that they will not overflow the page
% margins by default, and it is still possible to overwrite the defaults
% using explicit options in \includegraphics[width, height, ...]{}
\setkeys{Gin}{width=\maxwidth,height=\maxheight,keepaspectratio}
\IfFileExists{parskip.sty}{%
\usepackage{parskip}
}{% else
\setlength{\parindent}{0pt}
\setlength{\parskip}{6pt plus 2pt minus 1pt}
}
\setlength{\emergencystretch}{3em}  % prevent overfull lines
\providecommand{\tightlist}{%
  \setlength{\itemsep}{0pt}\setlength{\parskip}{0pt}}
\setcounter{secnumdepth}{0}
% Redefines (sub)paragraphs to behave more like sections
\ifx\paragraph\undefined\else
\let\oldparagraph\paragraph
\renewcommand{\paragraph}[1]{\oldparagraph{#1}\mbox{}}
\fi
\ifx\subparagraph\undefined\else
\let\oldsubparagraph\subparagraph
\renewcommand{\subparagraph}[1]{\oldsubparagraph{#1}\mbox{}}
\fi

%%% Use protect on footnotes to avoid problems with footnotes in titles
\let\rmarkdownfootnote\footnote%
\def\footnote{\protect\rmarkdownfootnote}

%%% Change title format to be more compact
\usepackage{titling}

% Create subtitle command for use in maketitle
\providecommand{\subtitle}[1]{
  \posttitle{
    \begin{center}\large#1\end{center}
    }
}

\setlength{\droptitle}{-2em}

  \title{Homework 0}
    \pretitle{\vspace{\droptitle}\centering\huge}
  \posttitle{\par}
  \subtitle{Warming up to R and R Markdown}
  \author{YOUR NAME HERE}
    \preauthor{\centering\large\emph}
  \postauthor{\par}
    \date{}
    \predate{}\postdate{}
  

\begin{document}
\maketitle

\textbf{Due before 2:30pm Monday January 13}

In this homework assignment, you will become more familiar with using
\texttt{R} and R Markdown to complete lab assignments.

First, make sure you have:

\begin{quote}
1a) Changed the \texttt{author} field in the YAML header to your name;
\end{quote}

\begin{quote}
1b) Clicked \texttt{Knit} to check that you can create a html document
from the .Rmd file;
\end{quote}

\begin{quote}
1c) Saved the .Rmd file as \texttt{LastnameFirstname-homework0.Rmd} in a
project named \texttt{LastNameFirstName-Lab0}
\end{quote}

If you have any problems with these steps, be sure to get help from
either the instructor or the TA

\hypertarget{r-markdown-basics}{%
\section{R Markdown basics}\label{r-markdown-basics}}

\hypertarget{basic-formatting}{%
\subsection{Basic formatting}\label{basic-formatting}}

\hypertarget{the-yaml-header}{%
\subsubsection{The YAML header}\label{the-yaml-header}}

At the top of your \texttt{.Rmd} file, you should see several line in
between three blue dashes:

\begin{verbatim}
---
title: "WILD3810: Homework 0"
subtitle: "Warming up to R and R Markdown"
author: "YOUR NAME HERE"
output: html_document
---
\end{verbatim}

This is called the ``YAML header'' and it's where we can control a lot
of the major formatting options for our documents. For example, to
change the output to PDF, just switch \texttt{html\_document} for
\texttt{pdf\_document} and then click the \texttt{Knit} button again (be
sure to change it back to html after you try it for yourself)

Pretty cool, right?

The YAML header allows to control many ``high level'' options for our
document. For example, to change the font size, type the following
directly under the \texttt{output:\ pdf\_document} argument:

\begin{verbatim}
fontsize: 12pt
\end{verbatim}

Check to see that the font size changed by clicking \texttt{Knit}.

Changing font type is a little trickier. Behind the scenes, R Markdown
turns your document into Latex code, which is then converted into a pdf.
You don't need to know much about Latex (though a little knowledge is
helpful) but this conversion does mean that our formatting options have
to passed to the Latex convertor in specific ways. To tell Latex that we
want to use \texttt{Arial} font, we have to modify the \texttt{output:}
argument as follows:

\begin{verbatim}
title: "WILD3810"
subtitle: "Homework 1"
author: "YOUR NAME HERE"
date: "2019-12-17"
output: 
  pdf_document:
    latex_engine: xelatex

mainfont: Arial
\end{verbatim}

Make sure you include the spaces to indent \texttt{pdf\_document:} and
\texttt{latex\_engine:\ xelatex}.

To indent the first line of each paragraph, add the following to the
header:

\begin{verbatim}
indent: true
\end{verbatim}

There many possible options for the header (see
\href{https://www.rstudio.com/wp-content/uploads/2016/03/rmarkdown-cheatsheet-2.0.pdf}{here}
for additional examples). We'll learn more about some of these options
later in the semester.

\hypertarget{headers}{%
\subsection{Headers}\label{headers}}

Using headers is a natural way to break up a document or report into
smaller sections. You can include headers by putting one or more
\texttt{\#} signs in front of text. One \texttt{\#} is a main header,
\texttt{\#\#} is the secondary header, etc.

\hypertarget{header-1}{%
\section{Header 1}\label{header-1}}

\hypertarget{header-2}{%
\subsection{Header 2}\label{header-2}}

\hypertarget{header-3}{%
\subsubsection{Header 3}\label{header-3}}

\hypertarget{paragraph-and-line-breaks}{%
\subsection{Paragraph and line breaks}\label{paragraph-and-line-breaks}}

When writing chunks of text in R Markdown (e.g., a report or
manuscript), you can create new paragraphs by leaving an empty line
between each paragraph:

\begin{verbatim}
This is one paragraph.

This is the next paragraph
\end{verbatim}

If you want to force a line break, include two spaces at the end of the
line where you want the break:

\begin{verbatim}
This is one line  
This is the next line
\end{verbatim}

\hypertarget{bold-italics}{%
\subsection{\texorpdfstring{\textbf{Bold},
\emph{Italics},}{Bold, Italics,}}\label{bold-italics}}

As mentioned earlier, create \textbf{boldface} by surrounding text with
two asterisks (\texttt{**bold**}) and use single asterisks for
\emph{italics} (\texttt{*italics*})

\hypertarget{code-type}{%
\subsection{Code type}\label{code-type}}

To highlight code (note, this does not actually insert functioning code,
just formats text to show that it is code rather than plain text),
surround the text with back ticks ": \texttt{mean()}

You can include muliple lines of code by including three back ticks on
the line before the code and then three back ticks on the line after the
code:

\begin{verbatim}
Multiple lines of code
look like 
this
\end{verbatim}

\hypertarget{bulleted-lists}{%
\subsection{Bulleted lists}\label{bulleted-lists}}

\begin{itemize}
\item
  Bulleted lists can be included by starting a line with an asterisk
\item
  You can also start the lines with a single dash \texttt{-}

  \begin{itemize}
  \tightlist
  \item
    for sub-bullets, indent the line and start it with \texttt{+}

    \begin{itemize}
    \tightlist
    \item
      for sub-sub-bullets, indent twice (press \texttt{tab} two times)
      and start with \texttt{-}
    \end{itemize}
  \end{itemize}
\end{itemize}

\hypertarget{numbered-lists}{%
\subsection{Numbered lists}\label{numbered-lists}}

\begin{enumerate}
\def\labelenumi{\arabic{enumi}.}
\item
  Numbered lists look like this
\item
  You can also include sub-levels in number lists

  \begin{enumerate}
  \def\labelenumii{\roman{enumii})}
  \tightlist
  \item
    these can be lower case roman numerals\\
  \end{enumerate}

  \begin{enumerate}
  \def\labelenumii{\alph{enumii}.}
  \tightlist
  \item
    or lowercase letters\\
    B. or uppercase letters
  \end{enumerate}
\end{enumerate}

\hypertarget{quotations}{%
\subsection{Quotations}\label{quotations}}

You highlight quotations by starting the line with
\texttt{\textgreater{}}, which produces:

\begin{quote}
All models are wrong
\end{quote}

\hypertarget{hyperlinks}{%
\subsection{Hyperlinks}\label{hyperlinks}}

Insert hyperlinks by putting the text you want displayed in square
brackets followed by the link in parentheses:
\texttt{{[}RStudio\ cheatsheet{]}(https://www.rstudio.com/wp-content/uploads/2016/03/rmarkdown-cheatsheet-2.0.pdf)}

\hypertarget{equations}{%
\subsection{Equations}\label{equations}}

Inserting equations in R Markdown is where knowing some Latex really
comes in handy becuase eqations are written using Latex code. For the
most part, this is not too difficult but if you need to insert complex
equations you will probably need to look up the code for some symbols.
There are many good resources for if you google ``latex equations'' or
something similear.

\hypertarget{inline-vs.block-equations}{%
\subsubsection{Inline vs.~block
equations}\label{inline-vs.block-equations}}

You can include equations either inline (\(e = mc^2\)) or as a
stand-alone block:

\[e=mc^2\]

Inline equations are added by putting a single dollar sign \texttt{\$}
on either side of the equation (\texttt{\$e=mc\^{}2\$}). Equation blocks
are create by starting and ending a new line with double dollar signs

\texttt{\$\$e=mc\^{}2\$\$}

\hypertarget{greek-letters}{%
\subsection{Greek letters}\label{greek-letters}}

Statistical models include a lot of Greek letters
(\(\alpha, \beta, \gamma\), etc.). You can add Greek letters to an
equation by typing a backslash \texttt{\textbackslash{}} followed by the
name of the letter \texttt{\textbackslash{}alpha}. Uppercase and lower
case letters are possible by capitalizing the name (\(\Delta\) =
\texttt{\$\textbackslash{}Delta\$}) or not (\(\delta\) =
\texttt{\$\textbackslash{}delta\$}).

\hypertarget{subscripts-and-superscripts}{%
\subsection{Subscripts and
superscripts}\label{subscripts-and-superscripts}}

You can add superscripts using the \texttt{\^{}}
(\(\pi r^2\)=\texttt{\$\textbackslash{}pi\ r\^{}2\$}) symbol and
subscripts using an underscore \texttt{\_} (\(N_t\) =
\texttt{\$N\_t\$}).

If the superscript or subscript includes more than one characeter, put
the entire script within curly brackets \texttt{\{\}}:
\(N_t-1 \neq N_{t-1}\) is
\texttt{\$N\_t-1\ \textbackslash{}neq\ N\_\{t-1\}\$}

\hypertarget{brackets-and-parentheses}{%
\subsection{Brackets and parentheses}\label{brackets-and-parentheses}}

You can add normal sized brackets and parenthesis just by typing them
into the equation: \((x + y)\) = \texttt{(x\ +\ y)}

If you need bigger sizes, using \texttt{\$\textbackslash{}big(\$},
\texttt{\$\textbackslash{}bigg(\$}, and
\texttt{\$\textbackslash{}Bigg(\$} produces \(\big(\), \(\bigg(\), and
\(\Bigg(\) (switch the opening parenthesis for a closing parenthesis or
square bracket as needed)

\hypertarget{fractions}{%
\subsection{Fractions}\label{fractions}}

Fractions can either be inline (\(1/n\) = \texttt{\$1/n\$}) or stacked
(\(\frac{1}{n}\) = \texttt{\$\textbackslash{}frac\{1\}\{n\}\$}). For
stacked equations, the terms in the first curly brackets are the
numerator and the terms in the second curly brackets are the
demoninator.

\hypertarget{operators}{%
\subsection{Operators}\label{operators}}

Pretty much every operator you could need can be written in latex. Some
common ones include \(\times\) (\texttt{\$\textbackslash{}times\$}),
\(\lt\) (\texttt{\$\textbackslash{}lt\$}), \(\gt\)
(\texttt{\$\textbackslash{}gt\$}), \(\leq\)
(\texttt{\$\textbackslash{}leq\$}), \(\geq\)
(\texttt{\$\textbackslash{}geq\$}), \(\neq\)
(\texttt{\$\textbackslash{}neq\$}), \(\sum\)
(\texttt{\$\textbackslash{}sum\$}), \(\prod\)
(\texttt{\$\textbackslash{}prod\$}), \(\infty\)
(\texttt{\$\textbackslash{}infty\$}), and \(\propto\)
(\texttt{\$\textbackslash{}propto\$}).

See \href{http://web.ift.uib.no/Teori/KURS/WRK/TeX/symALL.html}{here}
for a list of other operators.

\hypertarget{adding-code}{%
\section{Adding code}\label{adding-code}}

The ability to format and create pdf and html documents is great but the
real strength of R Markdown is the ability to include and run code
within your document. Code can be included \textbf{inline} or in
\textbf{chunks}

\hypertarget{inline-code}{%
\subsection{Inline code}\label{inline-code}}

Inline code is useful to including (simple) \texttt{R} output directly
into the text. Inline code can be added by inclosing \texttt{R} code
between \texttt{\textbackslash{}x60r} and \texttt{\textbackslash{}x60}.
For example, typing
\texttt{\textbackslash{}x60r\ mean(c(3,7,4,7,9))\textbackslash{}x60}
will compute and print the mean of the given vector. This is, it will
print 6 instead of the code itself. This can be very useful for
including summary statistics in reports.

For example, if we have a vector indicating the number of individuals
captured at each occassion during a mark-recapture study (e.g.,
\texttt{n\ \textless{}-\ c(155,\ 132,\ 147,\ 163)}) and we want to
include the number of occasions in a report, instead of typing
\texttt{4}, we can type
\texttt{\textbackslash{}x60r\ length(n)\textbackslash{}x60}. Not only
does this prevent typos, it is extemely useful if \texttt{length(n)}
might change in the future. Instead of manually changing the number of
occasions, we just re-render the document and the new number of
occasions will be printed automatically.

\hypertarget{code-chunks}{%
\subsection{Code chunks}\label{code-chunks}}

For more complicated code, it is generally more useful to use
\textbf{chunks} than inline code. Chunks start on a separate line with
\texttt{\textbackslash{}x60\textbackslash{}x60\textbackslash{}x60\{r\}}
and end with a
\texttt{\textbackslash{}x60\textbackslash{}x60\textbackslash{}x60} on
its own line (instead of doing this manually, you can click the
\texttt{Insert} button at the top right of script window, then click
\texttt{R}). In between these two lines, you can include as many lines
of code as you want. For example,

\begin{verbatim}
\x60\x60\x60{r}
n1 <- 44     # Number of individuals captured on first occasion

n2 <- 32     # Number of individuals captured on second occasion
  
m2 <- 15     # Number of previously marked individuals captured on second occasion

N <- n1 * n2 / m2     # Lincoln-Peterson estimate of abundance 
\x60\x60\x60
\end{verbatim}

\hypertarget{chunk-options}{%
\subsubsection{Chunk options}\label{chunk-options}}

Code chunks can take a lot of options to control how the code is run and
what is displayed in the documents. These options go after \texttt{\{r}
and before the closing \texttt{\}} (to see all the options put your
cursor after the \texttt{\{r}, hit the space bar, then hit
\texttt{tab}). For example:

\begin{itemize}
\item
  \texttt{echo\ =\ FALSE} shows the output of the code but not the code
  itself
\item
  \texttt{include\ =\ FALSE} runs the code but doe not display the code
  \emph{or} the output (useful for chunks that read or format data)
\item
  \texttt{eval\ =\ FALSE} shows the code but does not run it (useful for
  showing code)
\item
  \texttt{warning\ =\ FALSE} and \texttt{message\ =\ FALSE} can be
  include to ensure that error messages and warnings are not printed,
  which can be useful for cleaning up the appearance of documents
\item
  \texttt{cache\ =\ TRUE} save the results of the \texttt{R} code and
  doesn't rerun the chunk unless the code is changed (useful for chunks
  that take a long time to run)
\item
  \texttt{out.height} and \texttt{out.width} control the size of figures
  in a pdf document in inches or centimeters (e.g., `out.height =
  ``3in'', notice the quotation marks)
\end{itemize}

See the main \href{http://yihui.name/knitr/options/}{R Markdown page}
for a complete list of possible options.

\hypertarget{setting-defaults-for-all-chunks}{%
\subsubsection{Setting defaults for all
chunks}\label{setting-defaults-for-all-chunks}}

Often it is useful to set the default behavior for all chunks rather
that including, for example, \texttt{warning\ =\ FALSE} at the beginning
of each one. To do this, you can include a chunk at the beginning of the
document:

\begin{verbatim}
\x60\x60\x60{r include=FALSE}
opts_chunk$set(echo=FALSE, message=FALSE, warning=FALSE)
\x60\x60\x60
\end{verbatim}

Any options can be inluded in this chuck to set the default behaviors.
You can over-ride these defaults within chunks as needed. You can also
load common packages in this chunk to streamline chunks later in the
document.

\hypertarget{tables}{%
\subsubsection{Tables}\label{tables}}

To nicely print matrices and data frames in R Markdown document, use the
\texttt{kable()} function:

\begin{Shaded}
\begin{Highlighting}[]
\KeywordTok{library}\NormalTok{(knitr)}
\KeywordTok{kable}\NormalTok{(}\KeywordTok{head}\NormalTok{(mtcars))}
\end{Highlighting}
\end{Shaded}

\begin{longtable}[]{@{}lrrrrrrrrrrr@{}}
\toprule
& mpg & cyl & disp & hp & drat & wt & qsec & vs & am & gear &
carb\tabularnewline
\midrule
\endhead
Mazda RX4 & 21.0 & 6 & 160 & 110 & 3.90 & 2.620 & 16.46 & 0 & 1 & 4 &
4\tabularnewline
Mazda RX4 Wag & 21.0 & 6 & 160 & 110 & 3.90 & 2.875 & 17.02 & 0 & 1 & 4
& 4\tabularnewline
Datsun 710 & 22.8 & 4 & 108 & 93 & 3.85 & 2.320 & 18.61 & 1 & 1 & 4 &
1\tabularnewline
Hornet 4 Drive & 21.4 & 6 & 258 & 110 & 3.08 & 3.215 & 19.44 & 1 & 0 & 3
& 1\tabularnewline
Hornet Sportabout & 18.7 & 8 & 360 & 175 & 3.15 & 3.440 & 17.02 & 0 & 0
& 3 & 2\tabularnewline
Valiant & 18.1 & 6 & 225 & 105 & 2.76 & 3.460 & 20.22 & 1 & 0 & 3 &
1\tabularnewline
\bottomrule
\end{longtable}

The \texttt{kableExtra} package provides even more advanced options for
creating nice looking tables. See
\href{https://haozhu233.github.io/kableExtra/awesome_table_in_html.html}{here}
for an overview of options provided by this package.

\hypertarget{additional-resources}{%
\section{Additional resources}\label{additional-resources}}

From the RStudio tool bar, click
\texttt{Help\ -\textgreater{}\ Cheatsheets} and then select the
\texttt{R\ Markdown\ cheat\ sheet} (lots of other good cheat sheets
there as well)

RStudio's \href{https://rmarkdown.rstudio.com/lesson-1.html}{R Markdown}
tutorial

Tom Edward's \href{http://learnr.usu.edu/r_markdown/1_1_markdown.php}{R
Markdown} tutorial

Coding Club's
\href{https://ourcodingclub.github.io/2016/11/24/rmarkdown-1.html}{Getting
Started with R Markdown}

Cosma Shalizi's \href{http://www.stat.cmu.edu/~cshalizi/rmarkdown}{Using
R Markdown for Class Reports}

\hypertarget{homework-assignment-1}{%
\section{Homework assignment 1}\label{homework-assignment-1}}

Before submitting your assignment, be sure to double check the
instructions for
\href{https://rushinglab.github.io/WILD3810/articles/homework.html}{completing
and submitting homework assignments}.


\end{document}
